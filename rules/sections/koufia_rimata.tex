\section*{Ανώμαλα Ρήματα: κούφια}
Είναι τα ρήματα που στο \ar{ هوَ } του αορίστου έχουν \ar{ا} στην μέση και
που στον ενεστώτα μετρέπεται είτε σε \ar{ا}, είτε σε \ar{و} είτε σε \ar{ي}.
Είναι διπλόθεμα στον \textit{Αόριστο}.

\begin{enumerate}
\item Στο 3ο πρόσωπο (\ar{ هوَ ، هيَ  ، هُم }) το  θέμα είναι αυτό που δίνεται στο λεξολόγιο.
\item Σε όλα τα υπόλοιπα πρόσωπα:
\begin{enumerate}
\item παίρνουμε το θέμα 3ου προσώπου \ar{ كان }
\item αφαιρούμε το \ar{ ا } από την μέση $\rightarrow$ \ar{ كـــن }
\item βάζουμε ανάμεσά τους \textit{κάποιο} τόνο και προσθέτουμε τις καταλήξεις του αορίστου:

	\begin{enumerate}
	\item Αν το \ar{ ا } γίνεται \ar{و} τότε ο τόνος είναι \ar{ ــُـ }
	\item Αλλιώς (αν το \ar{ ا } γίνεται \ar{ي} ή μένει \ar{ا}) τότε ο τόνος είναι \ar{ ــِـ }
	\end{enumerate}

\end{enumerate}
\end{enumerate}

\begin{center}
\begin{tabular}{ c c c c c}
Αόριστος 1ο     & $\leftarrow$ & Ενεστώτας & Αόριστος & 3o \\
\ar{ انا قُمتُ }  & $\leftarrow$ & \ar{ يقومُ } & \ar{ قامَ }   & \ar{هوَ} \\
\ar{ انا نِمتُ }  & $\leftarrow$ & \ar{ ينامُ } & \ar{ نامَ }   & \ar{هوَ} \\
\ar{ انا عِشتُ }  & $\leftarrow$ & \ar{ يعيشُ } & \ar{ عاشَ }   & \ar{هوَ} \\
%\ar{ انا سِرتُ }   & $\leftarrow$ & \ar{ يسيرُ } & \ar{ سارَ } & \ar{هوَ} \\
%\ar{  }   & $\leftarrow$ & \ar{  } & \ar{  } \\
\end{tabular}
\end{center}


\begin{center}
\begin{tabular}{ c c c c }
       & Θέμα & \multicolumn{2}{c}{Αόριστος} \\
εγώ    & 2    & \ar{ زِلتُ }    & \ar{ انا }  \\
εσύ m  & 2    & \ar{ زِلتَ }    & \ar{ انتَ }  \\
εσύ f  & 2    & \ar{ زِلتِ }    & \ar{ انتِ }  \\
αυτός  & 1    & \ar{ زالَ }    & \ar{ هوَ }   \\
αυτή   & 1    & \ar{ زالَت }   & \ar{ هيَ }   \\
εμείς  & 2    & \ar{ زِلنا }   & \ar{ نَحنُ }  \\
εσείς  & 2    & \ar{ زِلتُم }   & \ar{ انتُم } \\
αυτοί  & 1    & \ar{ زالوا }  & \ar{ هُم }   \\
\end{tabular}
\end{center}
