\section*{Αριθμοί}

\begin{center}
\begin{tabular}{ c c c }
Νούμερα & Απόλυτα              & Τακτικά \\
1       & \ar{ واحِد ، واحِدة }  & \ar{ الاوّل ، الاُولى } \\
2       & \ar{ اِثنان ، اِثنَين } & \ar{ الثاني ، ة } \\
3       & \ar{ ثَلاثَ ، ة }       & \ar{ الثالِث ، ة } \\
4       & \ar{ اربَع ، ة }      & \ar{ الرابِع ، ة } \\
5       & \ar{ خَمسَ ، ة }       & \ar{ الخامِس ، ة } \\
6       & \ar{ سِتَّ ، ة }        & \ar{ السادِس ، ة } \\
7       & \ar{ سَبع ، ة }       & \ar{ السابِع ، ة } \\
8       & \ar{ ثَمانيَ ، ة }     & \ar{ الثامِن ، ة } \\
9       & \ar{ تِسع ، ة }       & \ar{ التاسِع ، ة } \\
10      & \ar{ عشْرَ ، عشَرة }    & \ar{ العاشِر ، ة } \\
11      & \ar{ احَد عشَرَ }       & \ar{ الحادي عشَر ، ة } \\
12      & \ar{ اِثنا عشَرَ }      & \ar{ الثاني عشَر ، ة } \\ \hline
13      & \ar{ ثَلاثة عشَرَ }      & \ar{ الثالِث عشَر ، ة } \\
14      & \ar{ اربَع عشَرَ }      & \ar{ الرابِع عشَر ، ة } \\
\dots   & \dots                & \dots \\
\end{tabular}
\end{center}

\subsection*{Δεκάδες}
προσθέτουμε από το τελευταίο σύμφωνο των μονάδων \ar{ ـونَ }. Εξαίρεση:
το 20 βγαίνει από το 10 \ar{ عَشرونَ }.

\subsection*{Σύνθεση}
Πρώτα λέμε τη μονάδα. Μετά βάζουμε \ar{و} ("και") και μετά βάζουμε τις δεκάδες:
\begin{center}
\begin{tabular}{ c c }
25  & \ar{خَمسَ و عشرونَ } \\
35  & \ar{خَمسَ و ثَلاثونَ } \\
\end{tabular}
\end{center}

\subsection*{Αριθμοί και ουσιαστικά}

\begin{description}
\item[1] χρησιμοποιείται ως επίθετο
\begin{center}
\begin{tabular}{ c c }
1 αδερφός  & \ar{ اخ واحِد } \\
1 αδερφή   & \ar{ اُخت واخِدة } \\
\end{tabular}
\end{center}

\item[2] 90\% με μορφή δυϊκού

\begin{center}
\begin{tabular}{ c c }
2 χρόνια    & \ar{ سَنَتين } \\
\end{tabular}
\end{center}

\item[3-10] \nl

	\begin{itemize}
	\item ουσιαστικό: μπαίνει στον πληθυντικό
	\item αριθμός: μπαίνει στο αντίθετο γένος από τον ενικό του ουσιαστικού.
	\end{itemize}

\begin{center}
\begin{tabular}{ c c }
3 φοιτητές    & \ar{ ثَلاثة طُلاب } \\
3 φοιτήτριες  & \ar{ ثَلاثَ طالِبات } \\
\end{tabular}
\end{center}

\item[11-99] \nl

\begin{itemize}
\item ουσιαστικό: ουσιαστικό μπαίνει στον ενικό
\item μονάδες: ισχύει ο κανόνας "3-10"
\end{itemize}

\begin{center}
\begin{tabular}{ c c }
45 φοιτητές    & \ar{ خمسة و اربعونَ طالِب } \\
45 φοιτήτριες  & \ar{ خَمسَ و اربعونَ طالِبة } \\
\end{tabular}
\end{center}

\end{description}

\section*{Ωρα}

\begin{center}
\begin{tabular}{ c c }
Η ώρα είναι                    & \ar{الساعة} \\
+ τακτικό αρ                   & + τακτικό αρ \textsuperscript{1}   \\
+ ( και | παρά )               & + ( \ar{إِلاّ} | \ar{و} )\\
+ ( τέταρτο | τρίτο | μισό ) | & + ( \ar{نِصف} | \ar{ثُلث} | \ar{رُبع} ) | \\
( απόλυτο αρ + λεπτά )         &  ( απόλυτο αρ + \ar{دَقيقة ، دَقائق} \textsuperscript{2} ) \\ \hline
ακριβώς                        & \ar{تاماماً} \\
\end{tabular}
\end{center}


Παραδείγματα:
\begin{center}
\begin{tabular}{ c c c }
1:15    & \ar{ الساعة الواحِدة و رُبع } &  \textsuperscript{1} \\
4:45    & \ar{ الساعة الحامِسة إِلاّ رُبع } &  \\
2:20    & \ar{ الساعة الثانية و ثُلث } &  \\
3:30    & \ar{ الساعة الثالِثة و نِصف } &  \\
6:55    & \ar{ الساعة السابعة إِلاّ خَمسَ دَقائق }  &  \textsuperscript{2} \\
7:10    & \ar{ الساعة السابعة و عشرَ دَقائق }   & \textsuperscript{2} \\
8:25    & \ar{ الساعة الثامِنة و خَمسَ و عشرونَ دَقيقة } &  \textsuperscript{2} \\
10:00   & \ar{ الساعة العشِرة تاماماً } &  \\
\end{tabular}
\end{center}

\textsuperscript{1} Για την 1:xx χρησιμοποιούμε το απόλυτο (\ar{واحِدة})

\textsuperscript{2} Προσοχή στους κανόνες αριθμών "3-10" και "11-99"
