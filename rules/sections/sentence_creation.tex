\section*{Δημιουργία προτάσεων}

\begin{enumerate}
\item Όταν η 1η λέξη έχει άρθρο και η 2η δεν έχει, τότε ανάμεσά τους στην \textbf{μετάφραση} μπαίνει το "είναι".

\begin{center}
\begin{tabular}{ c c }
το βιβλίο \tb{είναι} μεγάλο & \ar{ الكِتاب كَبير } \\
\end{tabular}
\end{center}

\item Όταν η 1η λέξη δεν έχει άρθρο και η 2η έχει, τότε ανάμεσά τους στην \textbf{μετάφραση} μπαίνει το "του" (γενική κτητική).

\begin{center}
\begin{tabular}{ c c }
το βιβλίο \tb{του} φοιτητή & \ar{ كِتاب الطالِب } \\
\end{tabular}
\end{center}

\item Επιθετικός προσιορισμός: το επίθετο μπαίνει μετά το ουσιαστικό. Ότι άρθρο παίνρει το ουσιαστικό παίρνει και το επίθετο.

\begin{center}
\begin{tabular}{ c c }
το μεγάλο βιβλίο (το βιλίο το μεγάλο) & \ar{ الكِتاب الكَبير } \\
\end{tabular}
\end{center}

\end{enumerate}
