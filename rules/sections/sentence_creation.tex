\section*{Δημιουργία προτάσεων}

\begin{enumerate}
\item Όταν η 1η λέξη είναι ορισμένη και η 2η δεν είναι, τότε ανάμεσά τους στην \textbf{μετάφραση} μπαίνει "είναι".

\begin{center}
\begin{tabular}{ c c }
το βιβλίο \tb{είναι} μεγάλο & \ar{ الكِتاب كَبير } \\
το βιβλίο μου \tb{είναι} μεγάλο & \ar{ كِتابي كَبير } \\
ο Μαχμούντ \tb{είναι} μεγάλος & \ar{ مَحمود كَبير } \\
\end{tabular}
\end{center}

\item Όταν η 1η λέξη δεν είναι ορισμένη και η 2η είναι, τότε ανάμεσά τους στην \textbf{μετάφραση} μπαίνει "του" (γενική κτητική).

\begin{center}
\begin{tabular}{ c c }
το βιβλίο \tb{του} φοιτητή & \ar{ كِتاب الطالِب } \\
το βιβλίο \tb{του} φοιτητή μου & \ar{ كِتاب طالِبي } \\
το βιβλίο \tb{του} Μαχμούντ & \ar{ كِتاب مَحمود } \\
\end{tabular}
\end{center}

\item Επιθετικός προσιορισμός: το επίθετο μπαίνει μετά το ουσιαστικό. Αν το ουσιαστικό είναι ορισμένο τότε παίρνει άρθρο και το επίθετο. Αν το ουσιαστικό δεν είναι ορισμένο τότε δεν παίρνει άρθρο το επίθετο.

\begin{center}
\begin{tabular}{ c c }
το μεγάλο βιβλίο (το βιβλίο το μεγάλο) & \ar{ الكِتاب الكَبير } \\
το μεγάλο βιβλίο μου (το βιβλίο μου το μεγάλο)  & \ar{ كِتابي الكَبير } \\
ένα μεγάλο βιβλίο (ένα βιβλίο μεγάλο)  & \ar{ كِتاب كَبير } \\
\end{tabular}
\end{center}

\end{enumerate}
