\section*{\ar{انْ ، انَّ ، اِنَّ}}

\begin{description}
\item[\ar{ انْ }] "αν"
Μόριο υποτακτικής. Μετά πάντα ρήμα απευθείας

\begin{center}
\begin{tabular}{ c c }
Θέλω να φάω & \ar{ استَطِاعَ ان اكُلَ } \\
\end{tabular}
\end{center}

\item[\ar{ انَّ }] "αννα"
οτι. Υπάρχουν ρήματα που συντάσονται με να ή οτι.
Μετά ποτέ ρήμα απυθείας. Κολλάει πάνω του η κτητική αντωνυμία.

\begin{center}
\begin{tabular}{ c c }
Θεωρώ οτι είσαι όμορφος               & \ar{ اظُنُ انَّكَ جَميل } \\
Θεωρώ οτι ο αδερφός μου είναι όμορφος & \ar{ اظُنُ انَّ اخي جَميل } \\
\end{tabular}
\end{center}


\item[\ar{ اِنَّ }] "ιννα"
Ισχύει οτι και με το \ar{ انَّ }.
Αλλά χρησιμοποείται αποκλειστικά με το λέω \ar{ قالَ ، يَقولُ اِنَّ }.
Εισάγει πλάγιο λόγο.

\end{description}
