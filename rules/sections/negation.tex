\section*{Άρνησεις}

\begin{center}
\begin{tabular}{ c c c c c }
Άρνηση Ενεστώτα  & $\leftarrow$ & ρήμα στον ενεστώτα & + & \ar{ لا }   \\
Άρνηση Αορίστου  & $\leftarrow$ & ρήμα στον ενεστώτα & + & \ar{ لَم }  \\
Άρνηση Αορίστου  & $\leftarrow$ & ρήμα στον αόριστο  & + & \ar{ ما }  \\
Άρνηση Μέλλοντα  & $\leftarrow$ & ρήμα στον ενεστώτα & + & \ar{ لَن }  \\
\end{tabular}
\end{center}

\begin{itemize}
\item To ρήμα που μπαίνει στον ενεστώτα, χάνει το \ar{ ن } εκεί που το χάνει και η υποτακτική.

\item Με το \ar{ لَم } το ρήμα χάνει τον τόνο (\ar{ ـْ }).
\begin{center}
\ar{ انا لَم اكتُبْ }
\end{center}

\item Με το \ar{ لَن } το ρήμα παίρνει \ar{ ـَ } όπως και στην υποτακτική.
\begin{center}
\ar{ انا لَن اكتُبَ }
\end{center}
\end{itemize}
