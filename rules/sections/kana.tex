\section*{\ar{كانَ ، يكُونُ}}
Η ανωμαλία του είναι οτι έχει διπλό θέμα στον αόριστο.

\begin{enumerate}
\item θέμα 3ου προσώπου: \ar{ كان }
\item
	\begin{enumerate}
	\item πέρνουμε το \ar{ كان } και αφαιρούμε το \ar{ ا } $\rightarrow$ \ar{ كن }
	\item βάζουμε ανάμεσά τους ένα τόνο
	\item τόνο: τον βρίσκουμε από τον ενεστώτα \ar{ كُن }
	\item προσθέτουμε τις καταλήξεις
	\end{enumerate}
\end{enumerate}

\begin{center}
\begin{tabular}{ r r c }
\ar{ كُنتُ }   &  \ar{ انا }   &  \\
\ar{ كُنتَ }   &  \ar{ انتَ  }  &  \\
\ar{ كُنتِ }   &  \ar{ انتِ  }  &  \\
\ar{ كانَ }   &  \ar{ هوَ }    & \textsuperscript{*} \\
\ar{ كانَت }  &  \ar{ هيَ  }   & \textsuperscript{*} \\
\ar{ كُنّا }   &  \ar{ نَحنُ  }  &  \\
\ar{ كُنتُم }  &  \ar{ انتُم  } &  \\
\ar{ كانوا } &  \ar{ هُم  }   & \textsuperscript{*} \\
\end{tabular}
\end{center}


\section*{Χρήσεις του \ar{كانَ}}
\begin{description}
\item [ήταν]

\item [είχα] \ar{ عِند }/\ar{ لي } + \ar{كانَ}

(το \ar{كانَ} ΔΕΝ κλίνεται)

\begin{center}
\begin{tabular}{ c c }
TODO  & \ar{  } \\
\end{tabular}
\end{center}


\item [Παρατατικός] ρήμα κλίνεται στον ενεστώτα + \ar{كانَ}

(το \ar{كانَ} κλίνεται)

\begin{center}
\begin{tabular}{ c c }
έγραφα  & \ar{ انا كُنتُ اكتُبُ } \\
\end{tabular}
\end{center}

\end{description}
