\section*{Ανώμαλα Ρήματα: με \ar{ى} στο τέλος (μακσούρα)}

Έχουν πολλούς τύπους. Ενας από αυτούς:

Αυτά που έχουν \ar{ى} στον αόριστο το οποίο γίνεται \ar{ي} στον ενεστώτα

\begin{description}
\item[Αόριστος]

	\begin{enumerate}
	\item[]
	\item Στο \ar{ هيَ ، هُم } χάνουν το \ar{ى}.
	\item Στα υπόλοιπα πρόσωπα το \ar{ى} γίνεται \ar{ي} και το προηγούμενο γράμμα πέρνει \ar{ ـَ }
	\item Το \ar{ هوَ } έχει \ar{ى} (έτσι δίνεται στο λεξιλόγιο).
	\end{enumerate}

\item[Ενεστώτας]

	\begin{enumerate}
	\item[]
	\item Στο \ar{ انتِ ، انتُم ، هُم } χάνουν το \ar{ي}.
	\end{enumerate}

\end{description}

\begin{center}
\begin{tabular}{ c c c p{2cm} c c }
       & \multicolumn{2}{c}{Ενεστώτας}                   &  & \multicolumn{2}{c}{Αόριστος} \\
       & \multicolumn{2}{c}{\ar{ يَمضي }}                 &  & \multicolumn{2}{c}{\ar{ مَضى }} \\
εγώ    &                     \ar{ امضي }   & \ar{ انا }  &  & \textsuperscript{2} \ar{ مَضَيتُ }  & \ar{ انا } \\
εσύ m  &                     \ar{ تَمضي }   & \ar{ انتَ }  &  & \textsuperscript{2} \ar{ مَضَيتَ }  & \ar{ انتَ }\\
εσύ f  &\textsuperscript{1 *}\ar{ تَمضينَ }  & \ar{ انتِ }  &  & \textsuperscript{2} \ar{ مَضَيتِ }  & \ar{ انتِ }\\
αυτός  &                     \ar{ يَمضي }   & \ar{ هوَ }   &  &                     \ar{ مَضى }   & \ar{ هوَ } \\
αυτή   &                     \ar{ تَمضي }   & \ar{ هيَ }   &  & \textsuperscript{1} \ar{ مَضَت }   & \ar{ هيَ }\\
εμείς  &                     \ar{ نَمضي }   & \ar{ نَحنُ }  &  & \textsuperscript{2} \ar{ مَضَينا } & \ar{ نَحنُ }\\
εσείς  & \textsuperscript{1} \ar{ تَمضونَ }  & \ar{ انتُم } &  & \textsuperscript{2} \ar{ مَضَيتُم } & \ar{ انتُم }\\
αυτοί  & \textsuperscript{1} \ar{ يَمضونَ }  & \ar{ هُم }   &  & \textsuperscript{1} \ar{ مَضَوا }  & \ar{ هُم }\\
\end{tabular}
\end{center}

\textsuperscript{*} το \ar{ ي } είναι της κατάληξης \ar{ ـينَ }.

Υπενθύμιση: το \ar{ ـَـيـ } προφέρεται εϊ: μαντεϊτου \ar{ مَضَيتُ }
