\documentclass[twocolumn,a4paper]{article}

% Compile with xelatex

% fonts
\usepackage[cm-default]{fontspec}
\setmainfont[Mapping=tex-text,Scale=1]{DejaVu Sans} % nice greek chars
\newfontfamily\arabicfont[Script=Arabic,Scale=1.5]{Droid Arabic Naskh} % nice arabic chars

%\usepackage{longtable} % Doesn't not work with \documentclass[twocolumns] - avoid
\usepackage{supertabular,booktabs}

\usepackage[a4paper, margin=1in]{geometry}

% TODO check: table row height?
\usepackage{array}
\usepackage{leading}

% different langiages: greek, arabic
\usepackage{polyglossia}
\setmainlanguage{english}
\setotherlanguage{arabic}
\newcommand{\ar}[1]{\textarabic{#1}}

% sign for nouns with plural
\newcommand{\pl}{\raisebox{0.15ex}{\footnotesize ◍}}
% ⊕ 🀆 🀱 % other interesting unicode symbols

% append normal plural to nouns
\newcommand{\normpl}[1]{\ar{ #1، ات }}

% sign for verbs with both present and past tense (verb full)
\newcommand{\vrf}{\raisebox{0.15ex}{\footnotesize ◉}}
% sign for verbs with present only
\newcommand{\vr}{\raisebox{0.15ex}{\footnotesize ◎}}

\begin{document}

% increase vertical space between rows
\renewcommand*{\arraystretch}{3}
% ensure twocolumns are spaced out so that they do not overlap
\setlength\tabcolsep{1pt}

\begin{supertabular}{ c c }

μεγάλος     & \ar{ كَبير } \\
μικρός      & \ar{ صَغير } \\
όμορφος     & \ar{ جَميل } \\
άσχημος     & \ar{ قَبيح } \\
παλιός      & \ar{ قَديم } \\
καινούργιος & \ar{ جَديد } \\
σπίτι       & \ar{ بَيت } \\
φίλος \pl   & \ar{ صَديق ، اصدِقاء } \\
φίλη \pl    & \ar{ صَديقة ، صَديقات } \\
φοιτητής \pl & \ar{ طالِب ، طُلاب } \\
βιβλίο \pl  & \ar{ كِتاب ، كُتُب } \\
καθηγητής   & \ar{ اُستاذ} \\
άντρας      & \ar{ رَجُل } \\
δουλειά     & \ar{ عَمَل } \\
σημειωματάρειο & \ar{ دَفتَر } \\
όνομα \pl   & \ar{ اِسم ، اسماء } \\
πόλη        & \ar{ مَدينة } \\
περιοχή     & \ar{ مِنطَقة } \\
χόμπυ       & \ar{ هِواية } \\
κολύμβηση   & \ar{ سِباحة } \\
χώρα        & \ar{ بَلَد } \\
γλώσσα      & \ar{ لُغة } \\
Αραβική     & \ar{ عرَبية } \\
και         & \ar{ وَ } \\
σχολείο \pl & \ar{ مَدرَسة ، مَدارِس } \\
μάθημα      & \ar{ دَرس } \\
κορίτσι \pl & \ar{ بِنت ، بَنات } \\
αδελφός     & \ar{ اخ } \\
αδελφή      & \ar{ اُخت } \\
μαμά        & \ar{ اُم} \\
μπαμπάς     & \ar{ اب } \\
οικογένεια  & \ar{ اُسرة } \\
κοντός      & \ar{ قَصير } \\
ψηλός/μακρύς & \ar{ طَويل } \\
λογοτεχνία  & \ar{ ادَب } \\
παν/στήμειο & \ar{ جامِعة } \\
γραφείο     & \ar{ مَكتَب } \\
βιβλιοθήκη  & \ar{ مَكتَبة } \\
ευγενικός   & \ar{ لَطيف } \\
κοντινός    & \ar{ قَريب } \\
κουρασμένος & \ar{ تعَبان } \\
κτίριο      & \ar{ بِناية } \\
δωμάτιο     & \ar{ غُرفة } \\
πατέρας     & \ar{ والِد} \\
μητέρα      & \ar{ والِدة } \\
κατοικώ \vrf& \ar{ سَكَنَ ، يَسكُنُ } \\
δουλεύω \vrf& \ar{ عمِلَ ، يَعمَلُ } \\
γράφω \vrf  & \ar{ كَتَبَ ، يَكتُبُ } \\
σπουδάζω \vrf & \ar{ دَرَسَ ، يَدرُسُ } \\
πίνω \vrf   & \ar{ شَرِبَ ، يَشرَبُ } \\
αυτός (δεικτικό) & \ar{ هَذا } \\
αυτή (δεικτικό)  & \ar{ هَذِهِ } \\
καφές       & \ar{ قَهوة } \\
γιατρός     & \ar{ دُكتور } \\
ώρα/ρολόϊ   & \ar{ ساعة} \\
σε          & \ar{ في } \\
το ίδιο     & \dots\ar{ نَفس ال } \\
τάξη        & \ar{ صَف } \\
τι/ποιό what & \ar{ ما } \\
τί (ρήματα) & \ar{ ماذا } \\
πώς         & \ar{ كَيفَ } \\
που         & \ar{ اينَ } \\
ποιός (άνθρωπο)& \ar{ مَن } \\
χαμάμ       & \ar{ حَمام } \\
πόρτα       & \ar{ باب } \\
δρόμος      & \ar{ شارِع } \\
ψωμί        & \ar{ خُبز } \\
καλός (άνθρωπος-φαγητό) & \ar{ طَيب } \\
τραπέζι     & \ar{ طاوِلة } \\
γλυκό       & \ar{ حَلوى } \\
σε αυτόν    & \ar{ فيها } \\
σε αυτήν    & \ar{ فيهِ } \\
γνωρίζω \vrf & \ar{ عرَفَ ، اعرِفُ  } \\
από         & \ar{ مِن } \\
αυτοκίνητο  & \ar{ سَيارة } \\
χάρτης      & \ar{ خَريطة } \\
ημέρα (journee) & \ar{ النَهار } \\
μεταφραστής     & \ar{ مُتَرجِم } \\
μετάφραση       & \ar{ تَرجَمة } \\
ασχολία/δουλειά & \ar{ شُغل } \\
απασχολημένος   & \ar{ مَشغول } \\
admission   & \ar{ القُبول } \\
μόνος/μοναδικός & \ar{ وَحيد } \\
υπάλληλος   & \ar{ موَظَف } \\
εξειδικευμένος  & \ar{ مُتَخَصِّص } \\
θείος (μητ) & \ar{ خال } \\
όπου        & \ar{ حَيثُ } \\
επίσης      & \ar{ ايضاً } \\
πάντα       & \ar{ دائماً } \\
όντως       & \ar{ فعلاً } \\
απόγευμα    & \ar{ المَساء } \\
πρωί        & \ar{ الصَباء } \\
σύντροφος☭/φίλος & \ar{ صاحِب } \\
πολιτεία    & \ar{ وِﻻية } \\
στην πραγματικότητα & \ar{ في الحَقيقة } \\
φωτογραφεία \pl & \ar{ صورة ، صوَر } \\
αξιωματικός \pl & \ar{ ضابِط ، ضُباط } \\
συγγενής \pl   & \ar{ قريب ، اقارِب } \\
σχολή \pl      & \ar{ كُلية ، كُليات } \\
κιθάρα         & \ar{ قيثارة } \\
τραίνο         & \ar{ قِطار } \\
προς (to)      & \ar{ إِلى } \\
όχι            & \ar{ لا } \\
ναι            & \ar{ نعم } \\
μαζί με        & \ar{ مع } \\
πεινασμένος    & \ar{ جوعان } \\
νέοι/παιδιά    & \ar{ شَباب } \\
βρε (2ο πρόσωπο) & \ar{ يا } \\
story          & \ar{ قِصة ، قِصَص } \\
ταινία         & \ar{ فيلم } \\
καλωσήρθατε    & \ar{ اهلاً وَ سَهلاً } \\
ερώτηση        & \ar{ سؤال } \\
τώρα           & \ar{ اﻵن } \\
γιός \pl       & \ar{ اِبن ، أبناء } \\
θείος (πατέρα) \pl & \ar{ عمّ ، أعمام } \\
στρατός \pl    & \ar{ جَيش ، جُيوش } \\
διδάσκω \vrf   & \ar{ دَرّسَ ، يُدَرِّسُ } \\
γράμμα \pl     & \ar{ رِسالة ، رَسائل } \\
ο σύζυγος \pl  & \ar{ زَوج ، أزواج } \\
η σύζυγος \pl  & \ar{ زَوجة ، زوَجات } \\
επιστήμη \pl   & \ar{ عِلم ، عُلوم } \\
πολιτική       & \ar{ سِّياسة } \\
πολιτικές επι/μες & \ar{ العُلوم السِّياسيّة } \\
παιδί \pl      & \ar{ وَلَد ، اولاد } \\
σόι \pl        & \ar{ عائلة ، عائلات  } \\
στρουμφάκια    & \ar{ سَنافِر } \\
γάτα \pl       & \ar{ قَطة ، قِطَط } \\
σκύλος \pl     & \ar{ كَلب ، كِلاب  } \\
αστραπή        & \ar{ صاعقة } \\
Τhundercats   & \ar{ القِطَط الصاعقة } \\
μήπως          & \ar{ هَل } \\
τίτλος/διεύθυνση & \ar{ عنوان } \\
άρρωστος       & \ar{ مَريض } \\
φαρδύς         & \ar{ واسِع } \\
δημ σχολείο    & \ar{ المَدرَسة الاِبتِدائّة } \\
αποστηθίζω \vrf & \ar{ حفِظَ ، احفَظُ } \\
θυμάμαι \vrf   & \ar{ تَذَكَّرَ ، اتَذَكَّرُ } \\
συνάδελφος|μαθητήςm\pl & \ar{ زَميل ، زُمَلاء } \\
συνάδελφος|μαθητήςf\pl & \ar{ زَميلة ، زَميلات } \\
ταξίδι         & \ar{ سَّفَر اِلى } \\
ταξιδευω \vrf  & \ar{ سافَرَ ، اُسافِرُ اِلى } \\
παιδική ηλικία & \ar{ طُّفولة } \\
άτομο \pl      & \ar{ فَرد ، افراد } \\
πριν           & \ar{ قَبلَ } \\
όλα/κάθε       & \ar{ كُل } \\
παρακολουθώ \vrf & \ar{ شاهَدَ ، يُشاهِدُ } \\
μιλώ \vrf       & \ar{ تَكَلَمَ ، يَتَكَلَمُ } \\
Μέση Ανατολή   & \ar{ الشَرق الاوسَط } \\
παππούς        & \ar{ جَدّ } \\
%γιαγιά         & \ar{  } \\
λέξη \pl       & \normpl{ كَلِمة } \\
διαγώνισμα     & \ar{ اِمتِحان } \\
δύσκολος       & \ar{ صعب } \\
εύκολος        & \ar{ سَهل } \\
πολύ (très)    & \ar{ جِداً } \\
τράπεζα        & \ar{ بَنك } \\
ιστορία        & \ar{ تاريخ } \\
τμήμα          & \ar{ قِسم } \\
κρύος          & \ar{ بارِد } \\
ζεστός         & \ar{ حار } \\
καιρός         & \ar{ جَو } \\
καιρός         & \ar{ طَقس } \\
εποχή \pl      & \ar{ فَصل ، فُصول } \\
άνοιξη         & \ar{ رَّبيع } \\
καλοκαίρι      & \ar{ صَّيف } \\
φθινώπορο      & \ar{ خَريف } \\
χειμώνας       & \ar{ شِّتاء } \\
υγρασία        & \ar{ رُّطوبة } \\
βαθμός         & \ar{ دَرَجة } \\
θερμοκρασία    & \ar{ دَرَجة الحَرارة } \\
κύριος         & \ar{ سَيد } \\
διαβάζω  \vrf   & \ar{ قَرأ ، يَقرأُ } \\
excuse me      & \ar{ مِن فَضلِك } \\
τρώω  \vrf      & \ar{ أكَلَ ، يأكُلُ } \\
γάλα           & \ar{ حَليب } \\
καταγωγή       & \ar{ مِن أصل } \\
αγαπώ \vr      & \ar{ ــ ، أُحِبُّ } \\
sometimes      & \ar{ احياناً } \\
traffic        & \ar{ اِزدِحام } \\
νιώθω \vrf     & \ar{ شعرَ ، اشعُرُ بِـ } \\
high           & \ar{ عالية } \\
semester       & \ar{ فَصل دِراسيّ } \\
only           & \ar{ فَقَط } \\
beaucoup       & \ar{ كَثيراً } \\
μοναξιά        & \ar{ وِحدة } \\
εξαιτίας + του & \ar{ بِسَبَب } \\
ο καλυτερος\&η & \ar{ احسَن } \\
εμπόριο        & \ar{ تِجارة } \\
αποκτώ \vrf    & \ar{ حَصَلَ ، يَحصُلُ على } \\
απόκτηση       & \ar{ حُصول على } \\
αποφοιτώ \vrf  & \ar{ تَخَرَجَ ، يَتَخَرَجُ مِن } \\
πηγαίνω \vrf   & \ar{ ذَهَبَ ، يَذهَبُ اِلى } \\
βδομάδα \pl    & \ar{ اُسبوع ، اسابيع } \\
έτος \pl       & \ar{ سَنة ، سَنَوات } \\
μέρα \pl       & \ar{ يَوم ، ايام } \\
σήμερα         & \ar{ اليَومَ } \\
ε\ldots (μτφ: και) & \ar{ فَــ } \\
κατά τη γνώμη+κτητ αντ & \ar{ بِالنِسبة لِـ } \\
διάλεξη \pl    & \normpl{ مُحاضَرة } \\
διοίκηση επιχ  & \ar{ اِدارة اﻷعمال } \\
that (αυτός/ή) & \ar{ ذَلِكَ ، تِلكَ } \\
λέκτορας \pl   & \ar{ مُعيد ، ون } \\
για/για να     & \ar{ لِـ } \\
γιατί;         & \ar{ لِماذا } \\
διότι          & \ar{ ﻷِنَّ } \\
γιαυτό         & \ar{ لِذَلِكَ } \\
εδώ και/πριν από & \ar{ مُنذُ } \\
εδώ και 2 χρόνια & \ar{ مُنذُ سَنَتين } \\
πρόγραμμα      & \ar{ بَرنامَج } \\
παρακολούθηση  & \ar{ مُشاهَدة } \\
Σάββατο        & \ar{ سَبت } \\
Κυριακή        & \ar{ احَد } \\
Δευτέρα        & \ar{ اِثنَين } \\
Τρίτη          & \ar{ ثُلاثاء } \\
Τετάρτη        & \ar{ اربِعاء } \\
Πέμπτη         & \ar{ خَميس } \\
Παρασκευή      & \ar{ جُمعة } \\
listen to (ακούω) \vrf & \ar{ اِستَمع ، يَستَمِعُ اِلى } \\
παράδειγμα     & \ar{ مِثال } \\
homework       & \ar{ واجِب } \\
τα νέα \pl     & \ar{ خَبَر ، اخبار } \\

το να πηγαίνω  & \ar{ ذَهاب اِلى } \\
το να αποστηθίζω & \ar{ حفِظ } \\
σπουδές        & \ar{ دِراسة } \\
έκθεση         & \ar{ كِتابة } \\
ανάγνωση       & \ar{ قِراءة } \\
αίσθημα        & \ar{ شُعور بِـ } \\
γνώση          & \ar{ معرِفة } \\
το να κατοικείς & \ar{ سَكَن } \\

το να πίνω     & \ar{ شُرب } \\
το να τρώω     & \ar{ اكل } \\
διδασκαλία     & \ar{ تَدريس } \\
αγάπη          & \ar{ حُب } \\
αποφοίτηση     & \ar{ تَخرُج } \\
το να θυμάμαι  & \ar{ تَذَكُر } \\
ομιλία/λόγος   & \ar{ كَلام } \\
το να ακούω    & \ar{ اِستماع اِلى  } \\

δίπλωμα        & \ar{ دِبلوم } \\
ειστιατόριο \pl & \ar{ مَطعم ، مَطاعِم } \\
πρώτη          & \ar{ اُولى } \\
λύκειο         & \ar{ المَدرَسة الثانَوية } \\
3η/απολητύριο λυκείου & \ar{ الثانَوية العامة } \\
παππούδες/πρόγονοι & \ar{ اجداد } \\
ατύχημα  \pl   & \ar{ حادِث ، حاوادِث } \\
γυμνάσιο       & \ar{ المَدرَسة الاِعدادية } \\
γενικό/δημόσιο & \ar{ عام } \\
ζεί            & \ar{ يعيشُ } \\
οικονομία      & \ar{ الاِقتِصاد } \\
μέγιστος/γηραιοτερος & \ar{ اكبَر } \\
μεγαλύτερός μου & \ar{ اكبَر هُم } \\
ήταν           & \ar{ كانَت } \\
πέθανε         & \ar{ ماتَت } \\
υπουργείο \pl  & \normpl{ وِزارة } \\

% & \ar{  } \\
% & \ar{  } \\

\end{supertabular}

\clearpage

\begin{supertabular}{ c c }
1              & \ar{ واحِد } \\
2              & \ar{ } \\
3              & \ar{ ثَلاثَ } \\
4              & \ar{ } \\
5              & \ar{ } \\
6              & \ar{ } \\
7              & \ar{ } \\
8              & \ar{ } \\
9              & \ar{ } \\
10             & \ar{ } \\
πρώτος         & \ar{ اَوَّل } \\
τέταρτος       & \ar{ رابِع } \\
\end{supertabular}

\clearpage


\begin{supertabular}{ c c }
Ισπανία      & \ar{ إسبانيا } \\
Πορτογαλλία  & \ar{ البُرتُغال } \\
Γαλλία       & \ar{ فَرانسا } \\
Γερμανία     & \ar{ المانيا } \\
Δανία        & \ar{  الدَنِمارك} \\
Ολλανδία     & \ar{ هولَندا } \\
Βέλγιο       & \ar{ بِلجيكا } \\

Ιταλία       & \ar{ إيطاليا } \\
Ελλάδα       & \ar{ اليونان } \\
Τουρκία      & \ar{ توركيا } \\

Ιρλανδία     & \ar{ أيرلَندا } \\
Ην. Βασίλειο & \ar{ المَملكة المُتَحِدة }\\

Συρία        & \ar{ سوريا } \\
Παλαιστήνη   & \ar{ فَلِسطين } \\
Λίβανος      & \ar{ لِبَنان } \\
Ιορδανία     & \ar{ الاُردُن } \\


Ηνωμένα Έθνη & \ar{ اﻻُمَم المُتَحِدة } \\
Αμερική      & \ar{ امريكا } \\
Ευρώπη       & \ar{ اوروبا } \\
Αυστραλία    & \ar{ استراليا } \\
Ασία         & \ar{ آسَيا } \\
Αφρική       & \ar{ أفريقيا } \\

Αίγυπτος     & \ar{ مِصر } \\
Λιβύη        & \ar{ ليبيا } \\
Τυνησία      & \ar{ تونيس } \\
Αλγερία      & \\  %\ar{ الجَزائر } \\
Μαρόκο       & \\ %\ar{ مَغرَب } \\
Σουδάν       & \ar{ سودان } \\


Κίνα         & \ar{ الصين } \\
Ινδία        & \ar{ الهِند } \\
Πακιστάν     & \ar{ البَكيستان } \\
% ?        & \ar{ اليابان } \\
\\
Καλιφόρνια   & \ar{ كاليفورنيا } \\
Νέα Υόρκη    & \ar{ نيويورك } \\
Αμστερνταμ   & \ar{ امستردام } \\
Κάιρο        & \ar{ القاهرة } \\
Δαμασκός     & \ar{ دِمَشق } \\
Λονδίνο      & \ar{ لندن } \\
Σαράγιεβο    & \ar{ سراييفو } \\
Τόκυο        & \ar{ طوكيو } \\
Πεκίνο       & \ar{ بيجين } \\
Σιδνευ       & \ar{ سيدني } \\
Βομβάη       & \ar{ مومباي } \\
Μανίλα       & \ar{ مانيلا } \\


% & \ar{  } \\
% & \ar{  } \\
% & \ar{  } \\
% & \ar{  } \\
% & \ar{  } \\

\end{supertabular}

\end{document}
